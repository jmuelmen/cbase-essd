%% $Id: method.tex,v 1.1 2015/11/15 10:24:32 jmuelmen Exp $

%% description of the retrieval method and the evaluation using various
%% ground-based observations

%% Copernicus Publications Manuscript Preparation Template for LaTeX Submissions
%% ---------------------------------
%% This template should be used for copernicus.cls
%% The class file and some style files are bundled in the Copernicus Latex Package which can be downloaded from the different journal webpages.
%% For further assistance please contact the Copernicus Publications at: publications@copernicus.org
%% http://publications.copernicus.org


%% Please use the following documentclass and Journal Abbreviations for Discussion Papers and Final Revised Papers.


%% 2-Column Papers and Discussion Papers
\documentclass[amt,manuscript]{copernicus}



%% Journal Abbreviations (Please use the same for Discussion Papers and Final Revised Papers)

% Archives Animal Breeding (aab)
% Atmospheric Chemistry and Physics (acp)
% Advances in Geosciences (adgeo)
% Advances in Statistical Climatology, Meteorology and Oceanography (ascmo)
% Annales Geophysicae (angeo)
% ASTRA Proceedings (ap)
% Atmospheric Measurement Techniques (amt)
% Advances in Radio Science (ars)
% Advances in Science and Research (asr)
% Biogeosciences (bg)
% Climate of the Past (cp)
% Drinking Water Engineering and Science (dwes)
% Earth System Dynamics (esd)
% Earth Surface Dynamics (esurf)
% Earth System Science Data (essd)
% Fossil Record (fr)
% Geographica Helvetica (gh)
% Geoscientific Instrumentation, Methods and Data Systems (gi)
% Geoscientific Model Development (gmd)
% Geothermal Energy Science (gtes)
% Hydrology and Earth System Sciences (hess)
% History of Geo- and Space Sciences (hgss)
% Journal of Sensors and Sensor Systems (jsss)
% Mechanical Sciences (ms)
% Natural Hazards and Earth System Sciences (nhess)
% Nonlinear Processes in Geophysics (npg)
% Ocean Science (os)
% Proceedings of the International Association of Hydrological Sciences (piahs)
% Primate Biology (pb)
% Scientific Drilling (sd)
% SOIL (soil)
% Solid Earth (se)
% The Cryosphere (tc)
% Web Ecology (we)



%% \usepackage commands included in the copernicus.cls:
%\usepackage[german, english]{babel}
%\usepackage{tabularx}
%\usepackage{cancel}
%\usepackage{multirow}
%\usepackage{supertabular}
%\usepackage{algorithmic}
%\usepackage{algorithm}
%\usepackage{amsthm}
%\usepackage{float}
%\usepackage{subfig}
%\usepackage{rotating}
\usepackage{mathptmx}
%% \usepackage[T1]{fontenc}

\begin{document}

%% \linenumbers

\title{Using CALIOP to estimate the base height of optically thick clouds}
%% \title{Cloud base height retrievals using 2B-GEOPROF-LIDAR and C-BREE$z$}
%% \title{Optimized Detection by Radar/lidAr for base of Nuages (ODR/AN)}


% \Author[affil]{given_name}{surname}

\Author[1]{Johannes}{M\"ulmenst\"adt}
\Author[1]{Odran}{Sourdeval}
\Author[1]{Edward}{Gryspeerdt}
\Author[1]{Johannes}{Quaas}

\affil[1]{Institute of Meteorology, Universit\"at Leipzig, Leipzig, Germany}
\affil[2]{Anyone else?}

%% The [] brackets identify the author with the corresponding affiliation. 1, 2, 3, etc. should be inserted.



\runningtitle{Cloud base height retrievals from CALIOP}

\runningauthor{M\"ulmenst\"adt et al.}

\correspondence{Johannes M\"ulmenst\"adt (\href{mailto:johannes.muelmenstaedt@uni-leipzig.de}{johannes.muelmenstaedt@uni-leipzig.de})}



\received{}
\pubdiscuss{} %% only important for two-stage journals
\revised{}
\accepted{}
\published{}

%% These dates will be inserted by Copernicus Publications during the typesetting process.


\firstpage{1}

\maketitle



\begin{abstract}
TEXT
\end{abstract}



\introduction  %% \introduction[modified heading if necessary]
\label{sec:intro}
Cloud base height is one of the most important parameters of a cloud.  It
controls how much downwelling longwave radiation the cloud emits.  Aerosol
concentration and updraft speed at that level control the microphysics of the
cloud.  It is one of the parameters that is necessary to calculate the
subadiabaticity of the cloud.  However, it is also one of the most difficult
parameters to retrieve from satellite.  Passive sensors: forget it.  CloudSat
misses the small droplets at the base and cannot retrieve in the ground clutter
region.  Calipso only detects the bases of only the thinnest clouds ($\tau < 5$,
according to Mace and Zhang, 10.1002/2013JD021374); frequently, these are not
the clouds you are looking for.

Because cloud base height varies slowly in space within an airmass, the cloud
bases retried by CALIOP for thin clouds may be a good proxy for the cloud base
heights of the entire cloud cluster, including the optically thicker clouds
within the cluster.  We have designed an algorithm that extrapolates the CALIOP
cloud-base measurements into locations where CALIOP attenuates before reaching
cloud base.  This algorithm is called CBASE (Cloud Base Altitude Spatial
Extrapolator).  In this paper we evaluate its performance by comparing CBASE
cloud base heights against cloud base heights observed by ground-based
ceilometers.

Section~\ref{sec:retrieval} describes the details of the retrieval algorithm.
In Section~\ref{sec:eval} we give a quantitative evaluation of the algorithm
including retrieval error statistics.  We conclude in
Section~\ref{sec:conclusions} with an outlook on the longstanding questions that
this retrieval can help address.

Literature: Meerk\"otter and Zinner (2007) (10.1029/2007GL030347) retrievals
using cloud top properties and adiabatic assumption; presumably there are
others.  Rosenfeld (2014) for cloud base temperature using the thinnest cloud in
a cloud cluster.  Mace and Zhang (2014) using CPR and CALIOP.  At least one more
using CloudSat; check Stephens et al (2012) energy-balance paper.

\section{Retrieval method}
\label{sec:retrieval}
Two sources of local cloud base: 
\begin{enumerate}
\item Caliop-Based Retrieval and Error Estimate of cloud-base $z$ (C-BREE$z$):
  \begin{itemize}
  \item Cloud top below 5 km MSL
  \item Cloud phase liquid or any
  \item CALIOP surface return indicates that the cloud is thin enough to trust
    the cloud base height
  \end{itemize}
\item 2B-GEOPROF-LIDAR method:
take the CALIPSO-only or CALIPSO+CloudSat retrievals of cloud base height.
\end{enumerate}

These cloud base retrievals only exist sporadically ($x$\% of columns), when
CALIOP happens to hit a sufficiently thin cloud.  To infer the cloud base height
$\hat{z}_i$ in a column $i$ away from the retrieval locations, we calculate the
weighted average of retrievals $z_j$ in columns $j$ within a given distance $D$
(specified in CloudSat footprints) of the desired location:
\begin{equation}
  \label{eq:z}
  \hat{z}_i = \frac{\sum\limits_{|i - j| < D} z_j w_{ij}}
  {\sum\limits_{|i - j| < D} w_{ij}}
\end{equation}
where the weight $w_{ij}$ given to each retrieval depends on the distance from
the desired position:
\begin{equation}
  \label{eq:w}
  w_{ij} = \left\{
    \begin{array}{cl}
      0 & \mbox{if no retrieval in column }j\\
      \frac{1}{\sqrt{2\pi\sigma^2}}\exp\left(-\frac12\frac{(i -
          j)^2}{\sigma^2}\right) & \mbox{otherwise}
    \end{array}\right.
\end{equation}
The accuracy of the $\hat{z}_i$ interpolated according to
(\ref{eq:z})--~(\ref{eq:w}) can be anticipated to decrease as the distance to
the available retrievals $z_j$ increases.  To be able to quantify this
dependence later on, we define a mean distance squared (MDS)
\begin{equation}
  \label{eq:d}
  \hat{d}^2_i = \frac{\sum\limits_{|i - j| < D} (i-j)^2 w_{ij}}
  {\sum\limits_{|i - j| < D} w_{ij}}
\end{equation}

Apart from the selection of the cloud base retrievals that form the input
for the interpolation, the algorithm has two tunable parameters.  The first is
the maximum distance from the desired location within which cloud base
retrievals will be used.  As long as sufficiently large values are chosen, this
parameter mainly governs computation time.  We have set this parameter to 200
CloudSat columns (approximately 280~km) to either side of the desired location
along the CloudSat track throughout the article.  The second tunable parameter
is the width of the Gaussian weighting function.  The choice of tunable
parameters is made on the basis of the evaluation
in Section~\ref{sec:eval-results}.  Once parameters have been chosen, the
evaluation is performed again on a \emph{different} evaluation data set to
ensure that the results can be generalized.

\subsection{Bias correction and uncertainty estimate}
In addition to MDS, the quality of the retrieval also depends on the homogeneity
of the cloud scene and the number of cloud layers, as will be shown in
Section~\ref{sec:eval-results}.  To account for this effect, the retrieval bias
as defined in Section~\ref{sec:eval-results} is calculated for different classes
of cloud fraction based on a subset of the evaluation data.  The retrieval is
then linearly corrected to eliminate the bias.  A new bias evaluation is
performed on a \emph{different} evaluation data set to ensure that the results
can be generalized.

\section{Evaluation method}
\label{sec:eval}
We assess the quality of the cloud base retrieval based on the following
metrics: 
\begin{description}
\item[Correlation coefficient] between the satellite retrieval and ground-based
  observation of the cloud base.  We use the Pearson correlation coefficient.
  Ideally the correlation coefficient would be unity.  
\item[Linear regression slope and intercept] (ideally 1 and 0, respectively).  
\item[RMS retrieval error,] defined as
  \begin{equation}
    \label{eq:rmse}
    \mbox{RMSE} = \frac{1}{N}\sqrt{\sum\limits_{i = 1}^{N}\left(\hat z_i - z_{\mathrm{obs}}\right)^2},
  \end{equation}
(ideally 0)
\item[Retrieval bias,] defined as
  \begin{equation}
    \label{eq:bias}
    \mbox{RMSE} = \frac{1}{N}\sqrt{\sum\limits_{i = 1}^{N}\left(\hat z_i - z_{\mathrm{obs}}\right)},
  \end{equation}(ideally 0)
\item[Efficiency,] i.e., probability that a retrieval is available at the
  desired location (ideally 1).
\end{description}
These metrics are calculated based on the set of all ground-based observations
for which a Calipso overpass is available and which meet the following
additional conditions:
\begin{itemize}
\item the ground-based cloud base height is below 3~km above ground level (AGL);
  this is the height to which the ceilometer-based aviation cloud base reports
  are reliable
\item for the evaluation of low-cloud retrievals, the ground-based cloud base
  height is below 3~km above mean sea level; this threshold height emulates the
  ISCCP definition of low cloud (cloud top pressure below 690~hPa)
\item for the evaluation of very low cloud retrievals, the ground-based cloud
  base height is below 1~km~AGL
\end{itemize}

The sources of ground-based observations are the following:
\begin{description}
\item[METARs] aviation routine and special weather reports (METARs) where the
  cloud base height is measured by ceilometer; to ensure that this is the case,
  we restrict ourselves to the contiguous continental United States.
\item[HD(CP)$^2$ ceilometers]
\item[ICOADS] (human observer ship-based observations)
\item[R/V \textit{Polarstern} ceilometer]
\item[MAGIC ceilometer]
\end{description}

\section{Evaluation results}
\label{sec:eval-results}
Figure~\ref{fig:metar-geoprof} shows  and Tbl.~blah.

\subsection{Variables controlling the retrieval quality}
distance from the closest CALIOP cloud-base retrieval, number of layers,
sky condition, cloud phase, terrain variability


\conclusions
\label{sec:conclusions}
\par Summary.  How useful is the cloud-base retrieval for various purposes?
There are two main cases: where accuracy is more important than efficiency, and
the other way around.  For surface radiative flux, accuracy is more important.
For cloud properties, efficiency.

\par 

\begin{acknowledgements}

\end{acknowledgements}

\begin{table}[t]
\caption{}
\vskip4mm
\centering
\begin{tabular}{p{\parindent}l|lll|lll}
\hline\hline
&& \multicolumn{3}{c|}{Local retrieval} & \multicolumn{3}{c}{C-BASE retrieval} \\
&& 2BGL & DARDAR & CAL & 2BGL & DARDAR & CAL \\
\hline
\multicolumn{2}{l|}{Efficiency} & & \\
& METAR & \\
& MAGIC & \\
& \chem{HD(CP)^2} & \\
\multicolumn{2}{l|}{Bias} & \\
\multicolumn{2}{l|}{RMSE} \\
\hline\hline
\end{tabular}
\end{table}

\begin{figure*}
  \centering
  \includegraphics[width=12cm, page=1]{2b-geoprof-lidar-2008.pdf}
  \caption{Comparison of the 2B-GEOPROF-LIDAR cloud-base retrieval with METARs}
  \label{fig:metar-geoprof}
\end{figure*}
\begin{figure}
  \centering
  \caption{Comparison of the 2B-GEOPROF-LIDAR cloud-base retrieval with METARs}
\end{figure}
\end{document}
