%% $Id: method.tex,v 1.1 2015/11/15 10:24:32 jmuelmen Exp $

%% description of the retrieval method and the evaluation using various
%% ground-based observations

%% Copernicus Publications Manuscript Preparation Template for LaTeX Submissions
%% ---------------------------------
%% This template should be used for copernicus.cls
%% The class file and some style files are bundled in the Copernicus Latex Package which can be downloaded from the different journal webpages.
%% For further assistance please contact the Copernicus Publications at: publications@copernicus.org
%% http://publications.copernicus.org


%% Please use the following documentclass and Journal Abbreviations for Discussion Papers and Final Revised Papers.


%% 2-Column Papers and Discussion Papers
\documentclass[amt,manuscript]{copernicus}\usepackage[]{graphicx}\usepackage[]{color}
%% maxwidth is the original width if it is less than linewidth
%% otherwise use linewidth (to make sure the graphics do not exceed the margin)
\makeatletter
\def\maxwidth{ %
  \ifdim\Gin@nat@width>\linewidth
    \linewidth
  \else
    \Gin@nat@width
  \fi
}
\makeatother

\definecolor{fgcolor}{rgb}{0.345, 0.345, 0.345}
\newcommand{\hlnum}[1]{\textcolor[rgb]{0.686,0.059,0.569}{#1}}%
\newcommand{\hlstr}[1]{\textcolor[rgb]{0.192,0.494,0.8}{#1}}%
\newcommand{\hlcom}[1]{\textcolor[rgb]{0.678,0.584,0.686}{\textit{#1}}}%
\newcommand{\hlopt}[1]{\textcolor[rgb]{0,0,0}{#1}}%
\newcommand{\hlstd}[1]{\textcolor[rgb]{0.345,0.345,0.345}{#1}}%
\newcommand{\hlkwa}[1]{\textcolor[rgb]{0.161,0.373,0.58}{\textbf{#1}}}%
\newcommand{\hlkwb}[1]{\textcolor[rgb]{0.69,0.353,0.396}{#1}}%
\newcommand{\hlkwc}[1]{\textcolor[rgb]{0.333,0.667,0.333}{#1}}%
\newcommand{\hlkwd}[1]{\textcolor[rgb]{0.737,0.353,0.396}{\textbf{#1}}}%
\let\hlipl\hlkwb

\usepackage{framed}
\makeatletter
\newenvironment{kframe}{%
 \def\at@end@of@kframe{}%
 \ifinner\ifhmode%
  \def\at@end@of@kframe{\end{minipage}}%
  \begin{minipage}{\columnwidth}%
 \fi\fi%
 \def\FrameCommand##1{\hskip\@totalleftmargin \hskip-\fboxsep
 \colorbox{shadecolor}{##1}\hskip-\fboxsep
     % There is no \\@totalrightmargin, so:
     \hskip-\linewidth \hskip-\@totalleftmargin \hskip\columnwidth}%
 \MakeFramed {\advance\hsize-\width
   \@totalleftmargin\z@ \linewidth\hsize
   \@setminipage}}%
 {\par\unskip\endMakeFramed%
 \at@end@of@kframe}
\makeatother

\definecolor{shadecolor}{rgb}{.97, .97, .97}
\definecolor{messagecolor}{rgb}{0, 0, 0}
\definecolor{warningcolor}{rgb}{1, 0, 1}
\definecolor{errorcolor}{rgb}{1, 0, 0}
\newenvironment{knitrout}{}{} % an empty environment to be redefined in TeX

\usepackage{alltt}



%% Journal Abbreviations (Please use the same for Discussion Papers and Final Revised Papers)

% Archives Animal Breeding (aab)
% Atmospheric Chemistry and Physics (acp)
% Advances in Geosciences (adgeo)
% Advances in Statistical Climatology, Meteorology and Oceanography (ascmo)
% Annales Geophysicae (angeo)
% ASTRA Proceedings (ap)
% Atmospheric Measurement Techniques (amt)
% Advances in Radio Science (ars)
% Advances in Science and Research (asr)
% Biogeosciences (bg)
% Climate of the Past (cp)
% Drinking Water Engineering and Science (dwes)
% Earth System Dynamics (esd)
% Earth Surface Dynamics (esurf)
% Earth System Science Data (essd)
% Fossil Record (fr)
% Geographica Helvetica (gh)
% Geoscientific Instrumentation, Methods and Data Systems (gi)
% Geoscientific Model Development (gmd)
% Geothermal Energy Science (gtes)
% Hydrology and Earth System Sciences (hess)
% History of Geo- and Space Sciences (hgss)
% Journal of Sensors and Sensor Systems (jsss)
% Mechanical Sciences (ms)
% Natural Hazards and Earth System Sciences (nhess)
% Nonlinear Processes in Geophysics (npg)
% Ocean Science (os)
% Proceedings of the International Association of Hydrological Sciences (piahs)
% Primate Biology (pb)
% Scientific Drilling (sd)
% SOIL (soil)
% Solid Earth (se)
% The Cryosphere (tc)
% Web Ecology (we)



%% \usepackage commands included in the copernicus.cls:
%\usepackage[german, english]{babel}
%\usepackage{tabularx}
%\usepackage{cancel}
%\usepackage{multirow}
%\usepackage{supertabular}
%\usepackage{algorithmic}
%\usepackage{algorithm}
%\usepackage{amsthm}
%\usepackage{float}
%\usepackage{subfig}
%\usepackage{rotating}
\usepackage{mathptmx}
%% \usepackage[T1]{fontenc}

\newcommand\comment[2]{\{\hlnum{ \textit{#1}: #2}\}}
\newcommand\commentjm[1]{\comment{$j_\mu$}{#1}}
\IfFileExists{upquote.sty}{\usepackage{upquote}}{}
\begin{document}



%% \linenumbers

\title{Using CALIOP to estimate the base height of optically thick clouds\\
%% \title{Cloud base height retrievals using 2B-GEOPROF-LIDAR and C-BREE$z$}
Used to be: ``Optimized Detection by Radar/lidAr for base of Nuages (ODR/AN)''}


% \Author[affil]{given_name}{surname}

\Author[1]{Johannes}{M\"ulmenst\"adt}
\Author[1]{Odran}{Sourdeval}
%%\Author[1]{Edward}{Gryspeerdt}
\Author[1]{Johannes}{Quaas}

\affil[1]{Institute of Meteorology, Universit\"at Leipzig, Leipzig, Germany}
\affil[2]{Anyone else?}

%% The [] brackets identify the author with the corresponding affiliation. 1, 2, 3, etc. should be inserted.



\runningtitle{Cloud base height retrievals from CALIOP}

\runningauthor{M\"ulmenst\"adt et al.}

\correspondence{Johannes M\"ulmenst\"adt (\href{mailto:johannes.muelmenstaedt@uni-leipzig.de}{johannes.muelmenstaedt@uni-leipzig.de})}



\received{}
\pubdiscuss{} %% only important for two-stage journals
\revised{}
\accepted{}
\published{}

%% These dates will be inserted by Copernicus Publications during the typesetting process.


\firstpage{1}

\maketitle



\begin{abstract}
  A measurement technique is presented that uses CALIOP lidar profiles to
  estimate cloud base heights.  The technique provides cloud base heights even
  when clouds are thick enough to attenuate the lidar beam (optical thickness
  $\tau \gtrsim 5$) by treating the cloud base height of nearby thinner clouds as
  representative of the entire cloud field.  Using ground-based ceilometer data,
  uncertainty estimates are derived as a function of various properties of the
  CALIOP lidar profiles.  Evaluation of the predicted cloud base heights and
  their predicted uncertainty using a second, statistically independent,
  ceilometer data set shows that cloud base heights and uncertainties are biased
  by less than 10\%.
\end{abstract}

\commentjm{General comments:
  \begin{itemize}
  \item switch to heights above ground $\longrightarrow$ done
  \end{itemize}
}

\introduction  %% \introduction[modified heading if necessary]
\label{sec:intro}
Cloud base height is one of the most important parameters of a cloud.  It
controls how much downwelling longwave radiation the cloud emits.  Aerosol
concentration and updraft speed at that level control the microphysics of the
cloud.  It is one of the parameters that is necessary to calculate the
subadiabaticity of the cloud.  However, it is also one of the most difficult
parameters to retrieve from satellite.  Oxygen absorption bands: planning to
evaluate these too.  VIIRS cloud-base temperature method: Zhu et al.~(2014, doi
10.1002/2013GL058970).  MISR stereoscopic imager method: working on it.
However, these are all not on the A-Train.  CloudSat misses the small droplets
at the base and cannot retrieve in the ground clutter region.  Calipso detects
the bases of only the thinnest clouds ($\tau < 5$, according to Mace and Zhang,
10.1002/2013JD021374); frequently, these are not the clouds you are looking for.

Because the lifting condensation level is approximately homogeneous within an
airmass, the cloud bases retried by CALIOP for thin clouds may be a good proxy
for the cloud base heights of the entire cloud cluster, including the optically
thicker clouds within the cluster.  We have designed an algorithm that
extrapolates the CALIOP cloud-base measurements into locations where CALIOP
attenuates before reaching cloud base.  This algorithm is called CBASE (Cloud
Base Altitude Spatial Extrapolator).  In this paper we evaluate its performance
by comparing CBASE cloud base heights against cloud base heights observed by
ground-based ceilometers.

Section~\ref{sec:data} describes the data sources used in determining and
evaluating the cloud base height.  In Section~\ref{sec:algorithm} we describe
the algorithm and evaluate its performance, including error statistics.  We
conclude in Section~\ref{sec:conclusions} with an outlook on the longstanding
questions that this retrieval can help address.

Literature: Meerk\"otter and Zinner (2007) (10.1029/2007GL030347) retrievals
using cloud top properties and adiabatic assumption; presumably there are
others.  Zhu et al.~(2014) for cloud base temperature using the thinnest cloud in
a cloud cluster.  Mace and Zhang (2014) using CPR and CALIOP.  At least one more
using CloudSat; check Stephens et al (2012) energy-balance paper.

\section{Data}
\label{sec:data}

Description of the CALIOP VFM data.

Description of the airport ceilometers.  METARs:
\verb+https://library.wmo.int/pmb_ged/wmo_49-v2_2013_en.pdf+; ASOS:
\verb+http://www.nws.noaa.gov/asos/pdfs/aum-toc.pdf+ 

\section{Algorithm and evaluation}
\label{sec:algorithm}

The algorithm and evaluation proceed in four steps:
\begin{enumerate}
\item We determine the cloud base height from all CALIOP profiles where the
  surface generates a return ($\longrightarrow$ lidar is not attenuated above
  cloud base).  We refer to these cloud base heights as \textit{local cloud base
    heights} in the sense that they are local to the CALIOP profile.
\item Using ground-based ceilometer data, we determine quality of cloud base
  height depending on a number of factors.  
\item Based on the predicted quality of each local cloud base, we either reject
  the local cloud base or combine it with other local cloud bases within a
  distance $D_\text{max}$ of the point of interest (POI) $\longrightarrow$ estimate of cloud base
  height, estimate of cloud base height uncertainty
\item Using a statistically independent validation dataset, we verify that the
  predicted cloud base height and uncertainty are correct.
\end{enumerate}

This section is divided into four subsections, one for each step enumerated
above.  Figure~\ref{fig:method} illustrates the method.

\subsection{Determination of local cloud base height}
\label{sec:algorithm:local}
Source of local cloud base from the CALIOP VFM: any profile with a surface
return.  

\subsection{Determination of local cloud base quality}
\label{sec:algorithm:qual}
We assess the quality of the CALIOP cloud base height $z$ using the
ceilometer-observed cloud base height $\hat{z}$ using the following
metrics: 
\begin{description}
\item[Correlation coefficient] between the satellite retrieval and ground-based
  observation of the cloud base.  We use the Pearson correlation coefficient.
  Ideally the correlation coefficient would be unity.  
\item[Linear regression slope and intercept] (ideally 1 and 0, respectively).  
\item[RMS retrieval error,] defined as
  \begin{equation}
    \label{eq:rmse}
    \mbox{RMSE} = \frac{1}{N}\sqrt{\sum\limits_{i = 1}^{N}\left(z_i - \hat{z}\right)^2},
  \end{equation}
  (ideally 0) \commentjm{This is the only metric that is actually used, so maybe
    we should just get rid of the rest.}
\item[Retrieval bias,] defined as
  \begin{equation}
    \label{eq:bias}
    \mbox{bias} = \frac{1}{N}\sum\limits_{i = 1}^{N}\left(z_i - \hat{z}\right),
  \end{equation}(ideally 0)
\item[Efficiency,] i.e., probability that a retrieval is available at the
  desired location (ideally 1).
\end{description}

These metrics are calculated based on the set of all ground-based observations
for which a Calipso overpass is available and which meet the following
additional conditions:
\begin{itemize}
\item the ground-based cloud base height is below 3~km above ground level (AGL);
  this is the height to which the ceilometer-based aviation cloud base reports
  are reliable
\item for the evaluation of low-cloud retrievals, the ground-based cloud base
  height is below 3~km above mean sea level; this threshold height emulates the
  ISCCP definition of low cloud (cloud top pressure below 690~hPa)
\end{itemize}

The sources of ground-based observations are the following:
\begin{description}
\item[METARs] aviation routine and special weather reports (METARs) where the
  cloud base height is measured by ceilometer; to ensure that this is the case,
  we restrict ourselves to the contiguous continental United States, where the
  cloud base height is mostly derived automatically by laser ceilometers that
  form part of Automated Surface Observing Stations (ASOS)
\item[HD(CP)$^2$ ceilometers] \commentjm{Unless there is an urgent reason, these
    will have to wait for a later paper}
\item[ICOADS] \commentjm{Unless there is an urgent reason, these
    will have to wait for a later paper}
\item[R/V \textit{Polarstern} ceilometer] \commentjm{Unfortunately, there is
    only a minute number of overpasses within 100~km, mostly at high latitudes}
\item[MAGIC ceilometer] \commentjm{Unfortunately, the number of overpasses here
    is also very small; need to check that I have all the MAGIC data}
\end{description}

\subsection{Combination of local cloud bases}
\label{sec:algorithm:combination}
These cloud base retrievals only exist sporadically (on average $x$\% of
columns), when CALIOP happens to hit a sufficiently thin cloud.  To infer the
cloud base height $z$ at a point of interest (POI) that does not necessarily
coincide with the location of a CALIOP profile, we proceed as follows.  We first
select all local CALIOP cloud base heights within a horizontal distance
$D_\text{max} = 100$~km of the POI.  On the basis of Section~\ref{sec:algorithm:qual}, we
discard cloud base heights unless
\begin{itemize}
\item VFM quality flag high confidence in the lowest cloud feature
\item lowest cloud feature is liquid (required because not enough data is
  available for reliable uncertainty prediction)
\item minimum horizontal averaging distance required for detection of the lowest
  cloud layer is 1~km or 1/3~km
\item geometric thickness of the lowest cloud layer is less than 1~km
\item the feature immediately above the surface is neither ``invalid'' nor ``no
  signal'' (indicating that although the surface may generate a detectable
  return, the lidar is sufficiently attenuated that the cloud base, which
  scatters less strongly than the surface, is unreliable)
\end{itemize}

For each remaining local cloud base height $z_i$, we determine the predicted
uncertainty $\sigma_i$ based on the categories established in the previous
section.  We determine a combined cloud base height
\begin{equation}
  \label{eq:combo-z}
  z = \frac{\sum\limits_i^n w_i z_i}{\sum\limits_i^n w_i}
\end{equation}
with weights
\begin{equation}
  \label{eq:weights}
  w_i = \frac 1 \sigma_i^2
\end{equation}
(optimal weights for uncorrelated least-squares).  In practice, the individual
measurements of cloud base are highly correlated with fairly similar
$\sigma_i$.  The cloud base estimate by Eq.~(\ref{eq:combo-z}) with weights
given by Eq.~(\ref{eq:weights}) remains unbiased even in the presence of
correlations.  However, for the combined cloud base uncertainty,
the uncorrelated weights would yield a biased estimate in the presence of
correlations.  The expression
\begin{equation}
  \label{eq:combo-sigma}
  \sigma^2 = \frac{\sum\limits_i^n \sigma_i^2}{n}
\end{equation}
yields acceptable results, as would be expected for highly correlated and fairly
similar $\sigma_i$.

\subsection{Evaluation of cloud base heights and cloud base height errors}
\label{sec:algorithm:eval}




Having tuned the algorithm on data from the year 2008, we evaluate it using a
statistically independent data set from the year 2007.  In the evaluation data
set, the ``true'' (i.e., measured by the ceilometer) cloud base height $\hat{z}$
is known in addition to the estimated cloud base height $z$ and the estimated
cloud base height uncertainty $\sigma$, determined according to the procedure
described in the previous section.

For satellite-derived measurements of the cloud base height $z$ that are
unbiased with respect to the ceilometer-observed cloud base heights $\hat{z}$
and have correctly estimated uncertainties $\sigma$, the pdf of the quantity
$(z - \hat{z})/\sigma$ has zero mean and unit standard deviation. In our
evaluation data set, we find a mean of 0.03 and a standard deviation of 1.05 -- so both the cloud base estimate and
the uncertainty estimate are unbiased at the better than 10\%\ level.

Discussion of representativeness of continental clouds over North America for
the remainder of the globe.  A marine validation dataset would be very welcome. 

\section{Results and data product availability}
\label{sec:results}

Geographic distributions of mean or median cloud base and thickness.  Annual
cycle (if it's interesting).  

Comparison with 2B-GEOPROF-LIDAR cloud bases: distinguish between radar-only and
lidar-only bases (both is rare for warm cloud).  Radar-only mean error is large
because the radar cloud base height predominantly clusters around the top of the
ground clutter region with little dependence on the actual cloud base height.
Lidar-only 2B-GEOPROF-LIDAR cloud base performs about as well as the CBASE cloud
base on average, but does not provide an uncertainty estimate (and therefore no
way to select only low-uncertainty cloud base estimates).

Data is available at DKRZ or PANGEA (need to look into which one is more
appropriate).  \commentjm{Obtain DOI, put it here.}

\conclusions
\label{sec:conclusions}
\par Summary.  How useful is the cloud-base retrieval for various purposes?
There are two main cases: where accuracy is more important than efficiency, and
the other way around.  For surface radiative flux (hyperlinear in CBT), accuracy
is more important.  For geometric thickness, efficiency.  For subadiabaticity,
need to do the error propagation.

\begin{acknowledgements}
CALIOP VFM from ICARE.  Computing (and data hosting?) at DKRZ.  Funding.
\end{acknowledgements}

\begin{table}[t]
\caption{}
\vskip4mm
\centering
\begin{tabular}{p{\parindent}l|lll|lll}
\hline\hline
&& \multicolumn{3}{c|}{Local retrieval} & \multicolumn{3}{c}{C-BASE retrieval} \\
&& 2BGL & DARDAR & CAL & 2BGL & DARDAR & CAL \\
\hline
\multicolumn{2}{l|}{Efficiency} & & \\
& METAR & \\
& MAGIC & \\
& \chem{HD(CP)^2} & \\
\multicolumn{2}{l|}{Bias} & \\
\multicolumn{2}{l|}{RMSE} \\
\hline\hline
\end{tabular}
\end{table}

\begin{figure}
  \centering
  \includegraphics[width=0.5\linewidth,keepaspectratio=true]{CloudFieldCALIOP.pdf}
  \caption{Schematic of CALIOP cloud base determination and evaluation strategy.
    \commentjm{Amazing figure by Christoph.  The only way to make it even better
      would be to add a ceilometer.}}
  \label{fig:method}
\end{figure}

\begin{figure*}
  \centering
\begin{knitrout}
\definecolor{shadecolor}{rgb}{0.969, 0.969, 0.969}\color{fgcolor}

{\centering \includegraphics[width=0.95\textwidth]{figure/method-eval-qual-1} 

}



\end{knitrout}
  \caption{Scatter plots of CALIOP versus ceilometer local cloud base height
    faceted by the CALIOP VFM QA flag}
  \label{fig:quality-qa}
\end{figure*}

\begin{figure*}
  \centering
\begin{knitrout}
\definecolor{shadecolor}{rgb}{0.969, 0.969, 0.969}\color{fgcolor}

{\centering \includegraphics[width=0.95\textwidth]{figure/method-combo-plot-1} 

}



\end{knitrout}
  \caption{Scatter plot of CBASE versus ceilometer cloud base height}
  \label{fig:eval}
\end{figure*}

\begin{figure}
  \centering
\begin{knitrout}
\definecolor{shadecolor}{rgb}{0.969, 0.969, 0.969}\color{fgcolor}

{\centering \includegraphics[width=0.5\textwidth]{figure/method-combo-eval-pull-1} 

}



\end{knitrout}
  \caption{Distribution of cloud base error divided by predicted uncertainty;
    least-squares gaussian fit with mean 0.03 and standard deviation 1.05 is overlaid.}
  \label{fig:pull}
\end{figure}

\begin{figure}
  \centering
\begin{knitrout}
\definecolor{shadecolor}{rgb}{0.969, 0.969, 0.969}\color{fgcolor}

{\centering \includegraphics[width=0.5\textwidth]{figure/method-combo-eval-rmse-1} 

}



\end{knitrout}
  \caption{Distribution of predicted uncertainty}
  \label{fig:uncertainty}
\end{figure}

\begin{figure*}
  \centering
\begin{knitrout}
\definecolor{shadecolor}{rgb}{0.969, 0.969, 0.969}\color{fgcolor}

{\centering \includegraphics[width=0.95\textwidth]{figure/method-combo-plot-rmseclass-1} 

}



\end{knitrout}
  \caption{Scatter plot of CBASE versus ceilometer cloud base height for
    different classes of predicted uncertainty}
  \label{fig:rmseclass}
\end{figure*}

\begin{table*}[t]
  \centering
  \caption{CBASE cloud base statistics by decile of predicted uncertainty}
  \label{tab:rmseclass}
% latex table generated in R 3.2.3 by xtable 1.8-3 package
% Sun May  7 13:49:30 2017
\begin{tabular}{lrrrrl}
  \hline
\hline
pred.rmse & $n$ & $r$ & RMSE (m) & bias (m) & fit \\ 
  \hline
(187,436] & 2624 & 0.749 & 403. & $-$33.2 & $y = 1.14 x - 63.5$ m \\ 
  (436,461] & 2624 & 0.722 & 429. & $-$26.4 & $y = 1.20 x - 186.$ m \\ 
  (461,478] & 2624 & 0.725 & 460. & $-$26.3 & $y = 1.25 x - 256.$ m \\ 
  (478,492] & 2624 & 0.693 & 466. & $-$29.4 & $y = 1.19 x - 182.$ m \\ 
  (492,506] & 2635 & 0.616 & 512. & $-$0.0279 & $y = 1.14 x - 157.$ m \\ 
  (506,517] & 2721 & 0.564 & 550. & $-$12.7 & $y = 1.14 x - 157.$ m \\ 
  (517,530] & 2516 & 0.566 & 562. & $-$0.809 & $y = 1.11 x - 131.$ m \\ 
  (530,550] & 2624 & 0.566 & 569. & $-$29.9 & $y = 1.15 x - 150.$ m \\ 
  (550,582] & 2624 & 0.497 & 640. & $-$11.8 & $y = 1.06 x - 66.1$ m \\ 
  (582,764] & 2624 & 0.446 & 716. & 13.1 & $y = 0.977 x + 14.6$ m \\ 
   \hline
\hline
\end{tabular}

\end{table*}

\begin{figure*}
  \centering

\begin{knitrout}
\definecolor{shadecolor}{rgb}{0.969, 0.969, 0.969}\color{fgcolor}

{\centering \includegraphics[width=0.95\textwidth]{figure/method-eval-2bgeoprof-1} 

}



\end{knitrout}
  %%<<eval-2bgeoprof-tbl,dev='tikz',fig.width=9.1,fig.height=4.5,out.width='\\textwidth',message=FALSE,cache=TRUE,echo=FALSE,results='asis'>>=
  %%@
  \caption{Scatter plot of 2B-GEOPROF-LIDAR versus ceilometer cloud base
    height.  \commentjm{I \textit{think} this uses Claudia's algorithm (minimum
      cloud base within $D_\text{max}$), but I have to check.}}
  \label{fig:eval-2b}
\end{figure*}

\begin{figure*}
  \centering
  %% <<vis-cbase,cache=TRUE,echo=FALSE,results='hide'>>=
  %% @
  %% <<vis-cbase2,dev='tikz',fig.width=7,fig.height=5,out.width='0.95\\textwidth',message=FALSE,cache=TRUE,echo=FALSE,results='hide'>>=
  %% @
  \caption{Geographic distribution of median cloud bases and median cloud
    thicknesses \commentjm{Make the figure less complex.  One panel with overall
    CBH.}}
  \label{fig:geo}
\end{figure*}




\end{document}
