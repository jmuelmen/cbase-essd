%% $Id$

%% description of the retrieval method and the evaluation using various
%% ground-based observations

\documentclass{article}
\usepackage{fullpage}

\setlength{\parindent}{0pt}

\begin{document}
\section{Introduction}
\label{sec:intro}
Cloud base height is one of the most important parameters of a cloud.  It
controls how much downwelling longwave radiation the cloud emits.  Aerosol
concentration and updraft speed at that level control the microphysics of the
cloud.  It is one of the parameters that is necessary to calculate the
subadiabaticity of the cloud.  However, it is also one of the most difficult
parameters to retrieve from satellite.  Passive sensors: forget it.  CloudSat
misses the small droplets at the base and cannot retrieve in the ground clutter
region.  Calipso only detects the bases of only the thinnest clouds ($\tau < 5$,
according to Mace and Zhang, 10.1002/2013JD021374); frequently, these are not
the clouds you are looking for.

Because cloud base height varies slowly in space within an airmass, the cloud
bases retried by CALIOP for thin clouds may be a good proxy for the cloud base
heights of the entire cloud cluster, including the optically thicker clouds
within the cluster.  In this paper we investigate this hypothesis by evaluating
cloud base heights thus inferred against cloud base heights retrieved by
ground-based ceilometers.  

Section~\ref{sec:retrieval} describes the details of the retrieval algorithm.
In Section~\ref{sec:eval} we give a quantitative evaluation of the algorithm
including retrieval error statistics.  We conclude in
Section~\ref{sec:conclusions} with an outlook on the longstanding questions that
this retrieval can help address.

Literature: Meerk\"otter and Zinner (2007) (10.1029/2007GL030347) retrievals
using cloud top properties and adiabatic assumption; presumably there are
others.  Rosenfeld (2014) for cloud base temperature using the thinnest cloud in
a cloud cluster.  Mace and Zhang (2014) using CPR and CALIOP.  At least one more
using CloudSat; check Stephens et al (2012) energy-balance paper.

\section{Retrieval method}
\label{sec:retrieval}
Two sources of local cloud base: 
\begin{enumerate}
\item DARDAR method:
  \begin{itemize}
  \item Cloud top below 5 km MSL
  \item Cloud phase liquid or any
  \item CALIOP surface return indicates that the cloud is thin enough to trust
    the cloud base height
  \end{itemize}
\item 2B-GEOPROF-LIDAR method:
take the CALIPSO-only or CALIPSO+CloudSat retrievals of cloud base height.
\end{enumerate}

These cloud base retrievals only exist sporadically ($x$\% of columns), when
CALIOP happens to hit a sufficiently thin cloud.  To infer the cloud base height
$\hat{z}_i$ in a column $i$ away from the retrieval locations, we calculate the
weighted average of retrievals $z_j$ in columns $j$ within a given distance $N$
(specified in CloudSat footprints) of the desired location:
\begin{equation}
  \label{eq:z}
  \hat{z}_i = \frac{\sum\limits_{|i - j| < N} z_j w_{ij}}
  {\sum\limits_{|i - j| < N} w_{ij}}
\end{equation}
where the weight $w_{ij}$ given to each retrieval depends on the distance from
the desired position:
\begin{equation}
  \label{eq:w}
  w_{ij} = \left\{
    \begin{array}{cl}
      0 & \mbox{if no retrieval in column }j\\
      \frac{1}{\sqrt{2\pi\sigma^2}}\exp\left(-\frac12\frac{(i -
          j)^2}{\sigma^2}\right) & \mbox{otherwise}
    \end{array}\right.
\end{equation}
The accuracy of the $\hat{z}_i$ interpolated according to
(\ref{eq:z})--~(\ref{eq:w}) can be anticipated to decrease as the distance to
the available retrievals $z_j$ increases.  To be able to quantify this
dependence later on, we define a mean distance squared (MDS)
\begin{equation}
  \label{eq:d}
  \hat{d}^2_i = \frac{\sum\limits_{|i - j| < N} (i-j)^2 w_{ij}}
  {\sum\limits_{|i - j| < N} w_{ij}}
\end{equation}

Apart from the selection of the cloud base retrievals that form the input
for the interpolation, the algorithm has two tunable parameters.  The first is
the maximum distance from the desired location within which cloud base
retrievals will be used.  As long as sufficiently large values are chosen, this
parameter mainly governs computation time.  We have set this parameter to 200
CloudSat columns (approximately 280~km) to either side of the desired location
along the CloudSat track throughout the article.  The second tunable parameter
is the width of the Gaussian weighting function.  The choice of tunable
parameters is made on the basis of the evaluation
in Section~\ref{sec:eval-results}.  Once parameters have been chosen, the
evaluation is performed again on a \emph{different} evaluation data set to
ensure that the results can be generalized.

\subsection{Bias correction}
In addition to MDS, the quality of the retrieval also depends on the homogeneity
of the cloud scene, as will be shown in Section~\ref{sec:eval-results}.  To
account for this effect, the retrieval bias as defined in
Section~\ref{sec:eval-results} is calculated for different classes of cloud
fraction based on a subset of the evaluation data.  The retrieval is then
linearly corrected to eliminate the bias.  A new bias evaluation is performed on
a \emph{different} evaluation data set to ensure that the results can be
generalized. 

\section{Evaluation method}
\label{sec:eval}
\par Evaluation metrics:
\begin{itemize}
\item correlation coefficient between satellite retrieval and ground-based
  observation (ideally 1)
\item regression slope (ideally 1) and intercept (ideally 0) between satellite
  retrieval and ground-based observation
\item RMS retrieval error (ideally 0)
\item retrieval bias (ideally 0)
\item ``efficiency'', i.e., probability of retrieval where you want it
\end{itemize}

\par Evaluation datasets:
\begin{itemize}
\item METARs (ceilometer cloud bases)
\item ICOADS (mixture of ceilometer and human observer)
\item Polarstern (ceilometer)
\item MAGIC (ceilometer)
\end{itemize}

\section{Evaluation results}
\label{sec:eval-results}

\subsection{Variables controlling the retrieval quality}



\section{Conclusions}
\label{sec:conclusions}
\par Summary.  How useful is the cloud-base retrieval for various purposes?
There are two main cases: where accuracy is more important than efficiency, and
the other way around.  For surface radiative flux, accuracy is more important.
For cloud properties, efficiency.

\par 
\end{document}
