%% description of the retrieval method and the evaluation using various
%% ground-based observations

%% Copernicus Publications Manuscript Preparation Template for LaTeX Submissions
%% ---------------------------------
%% This template should be used for copernicus.cls
%% The class file and some style files are bundled in the Copernicus Latex Package which can be downloaded from the different journal webpages.
%% For further assistance please contact the Copernicus Publications at: publications@copernicus.org
%% http://publications.copernicus.org


%% Please use the following documentclass and Journal Abbreviations for Discussion Papers and Final Revised Papers.


%% 2-Column Papers and Discussion Papers
\documentclass[essd,manuscript]{copernicus}
%% Journal Abbreviations (Please use the same for Discussion Papers and Final Revised Papers)

% Archives Animal Breeding (aab)
% Atmospheric Chemistry and Physics (acp)
% Advances in Geosciences (adgeo)
% Advances in Statistical Climatology, Meteorology and Oceanography (ascmo)
% Annales Geophysicae (angeo)
% ASTRA Proceedings (ap)
% Atmospheric Measurement Techniques (amt)
% Advances in Radio Science (ars)
% Advances in Science and Research (asr)
% Biogeosciences (bg)
% Climate of the Past (cp)
% Drinking Water Engineering and Science (dwes)
% Earth System Dynamics (esd)
% Earth Surface Dynamics (esurf)
% Earth System Science Data (essd)
% Fossil Record (fr)
% Geographica Helvetica (gh)
% Geoscientific Instrumentation, Methods and Data Systems (gi)
% Geoscientific Model Development (gmd)
% Geothermal Energy Science (gtes)
% Hydrology and Earth System Sciences (hess)
% History of Geo- and Space Sciences (hgss)
% Journal of Sensors and Sensor Systems (jsss)
% Mechanical Sciences (ms)
% Natural Hazards and Earth System Sciences (nhess)
% Nonlinear Processes in Geophysics (npg)
% Ocean Science (os)
% Proceedings of the International Association of Hydrological Sciences (piahs)
% Primate Biology (pb)
% Scientific Drilling (sd)
% SOIL (soil)
% Solid Earth (se)
% The Cryosphere (tc)
% Web Ecology (we)



%% \usepackage commands included in the copernicus.cls:
%\usepackage[german, english]{babel}
%\usepackage{tabularx}
%\usepackage{cancel}
%\usepackage{multirow}
%\usepackage{supertabular}
%\usepackage{algorithmic}
%\usepackage{algorithm}
%\usepackage{amsthm}
%\usepackage{float}
%\usepackage{subfig}
%\usepackage{rotating}
\usepackage{mathptmx}
%% \usepackage[T1]{fontenc}

%% The following will not appear in the final paper, since it will be resolved
%% during proofreading/typesetting:
\usepackage[textsize=11pt]{todonotes}
\usepackage{wasysym}
\newcommand{\commentjm}[1]{\todo[inline, color=red!50]{$j_\mu$: #1}}

\begin{document}



%% \linenumbers

\title{Using CALIOP to estimate cloud-field base height and its uncertainty: the
  Cloud Base Altitude Spatial Extrapolator (CBASE) algorithm and dataset}


% \Author[affil]{given_name}{surname}

\Author[1]{Johannes}{M\"ulmenst\"adt}
\Author[1]{Odran}{Sourdeval}
\Author[2]{David S.}{Henderson}
\Author[2]{Tristan S.}{L'Ecuyer}
\Author[1]{Claudia}{Unglaub}
\Author[1]{Leonore}{Jungandreas}
\Author[3]{Christoph}{B\"ohm}
\Author[4]{Lynn M.}{Russell}
\Author[1]{Johannes}{Quaas}

\affil[1]{Institute of Meteorology, Universit\"at Leipzig, Leipzig, Germany}
\affil[2]{University of Wisconsin at Madison, Madison, Wisconsin, USA}
\affil[3]{Institute for Geophysics and Meteorology, Universit\"at zu K\"oln,
  K\"oln, Germany}
\affil[4]{Scripps Institution of Oceanography, University of California, San
  Diego, San Diego, California, USA}

%% The [] brackets identify the author with the corresponding affiliation. 1, 2, 3, etc. should be inserted.



\runningtitle{Cloud base heights from CALIOP}

\runningauthor{M\"ulmenst\"adt et al.}

\correspondence{Johannes M\"ulmenst\"adt
  (\href{mailto:johannes.muelmenstaedt@uni-leipzig.de}{johannes.muelmenstaedt@uni-leipzig.de})}



\received{}
\pubdiscuss{} %% only important for two-stage journals
\revised{}
\accepted{}
\published{}

%% These dates will be inserted by Copernicus Publications during the typesetting process.


\firstpage{1}

\maketitle



\begin{abstract}
  A technique is presented that uses attenuated backscatter profiles from the
  CALIOP satellite lidar to estimate cloud base heights of lower-troposphere
  liquid clouds (cloud base height below approximately 3~\unit{km}).  Even when clouds are
  thick enough to attenuate the lidar beam (optical thickness $\tau \gtrsim 5$),
  the technique provides cloud base heights by treating the cloud base height of
  nearby thinner clouds as representative of the surrounding cloud field.  Using
  ground-based ceilometer data, uncertainty estimates for the cloud base height
  product at retrieval resolution are derived as a function of various
  properties of the CALIOP lidar profiles.  Evaluation of the predicted cloud
  base heights and their predicted uncertainty using a second, statistically
  independent, ceilometer dataset shows that cloud base heights and
  uncertainties are biased by less than 10\%.  Geographic distributions of cloud
  base height and its uncertainty are presented.  In some regions, the
  uncertainty is found to be substantially smaller than the 480~\unit{m}
  uncertainty assumed in the A-Train surface downwelling longwave estimate,
  potentially permitting the most uncertain of the radiative fluxes in the
  climate system to be better constrained.  The cloud base dataset is available
  at \url{https://doi.org/10.1594/WDCC/CBASE}.
\end{abstract}

\introduction  %% \introduction[modified heading if necessary]
\label{sec:intro}
The base height \ensuremath{z}{} is an important geometric parameter of a cloud,
controlling the cloud's longwave radiative emission, being required in the
calculation of the cloud's subadiabaticity, and setting the level at which
aerosol concentration and updraft speed determine the cloud's microphysical
characteristics.  However, due to the viewing geometry, it is also one of the
most difficult cloud parameters to retrieve from satellite.

Multiple methods have been proposed for satellite determination of the cloud
base height.  \cite{Zhu2014} have used the Visible Infrared Imaging Radiometer Suite
aboard the Suomi National Polar-orbiting Partnership satellite
\citep[VIIRS,][]{Cao2014} to estimate cloud base temperature $T_b$ from the
lowest cloud top temperature within a cloud cluster; a reanalysis temperature
profile can be used to convert $T_b$ to \ensuremath{z}{}.  Using an empirical relationship
between geometric and optical thickness, \cite{Fitch2016} have obtained \ensuremath{z}{} from
VIIRS.  Cloud geometric thickness (and therefore \ensuremath{z}{} if the cloud top height is
known) can be inferred from increased spectral absorption by \chem{O_2} within cloud
due to multiple scattering \citep{Kokhanovsky2005,Lelli2018}.  Stereoscopic determination of
the height of the most reflective layer \citep{Naud2005,Naud2007} in Multiangle
Imaging Spectroradiometer data \citep[MISR,][]{Diner1998} yields information on
\ensuremath{z}{}, as the lowest layer heights within a cloud cluster may correspond to the
base of a cloud seen from its side.  An evaluation of MISR
techniques is in progress \citep{Boehm2017}.
  
For analyses wishing to combine cloud base information with other cloud
properties retrieved by A-Train satellites, these methods share the
disadvantage that the required instruments are not part of the A-Train.
Methods that are applicable to A-Train satellites are based on
MODerate-resolution Imaging Spectroradiometer \citep[MODIS,][]{Platnick2017}
cloud properties retrieved near cloud top and integrated along moist adiabats
to determine the cloud thickness \citep{Meerkoetter2007,Goren2018} or on active remote
sensing by CloudSat \citep[2B-GEOPROF,][]{Marchand2008} or a combination of
CloudSat and CALIOP \citep[2B-GEOPROF-LIDAR,][]{Mace2014}.  Each of these has
drawbacks.  The MODIS-derived cloud thickness assumes adiabatic cloud profiles
and therefore cannot be used to constrain subadiabaticity; the use of
ancillary temperature profile estimates may also be problematic in many cases.
CloudSat misses the small droplets at the base of nonprecipitating clouds
\citep{Sassen2008}, and retrievals are further degraded in the ground clutter
region \citep{Tanelli2008, Marchand2008}.  CALIOP detects the bases of only
the thinnest clouds \citep[$\tau < 5$,][]{Mace2014}; frequently, it is
desirable to know the base height of thick clouds as well.

In this paper, we revisit the CALIOP cloud base determination.  We rely on
one central assumption, namely that, because the lifting condensation level is
approximately homogeneous within an airmass, the cloud bases retrieved by CALIOP
for thin clouds is a good proxy for the cloud base heights of an entire
cloud field, including the optically thicker clouds within the field.  We have
designed an algorithm that extrapolates the CALIOP cloud base measurements into
locations where CALIOP attenuates before reaching cloud base.  This algorithm is
called Cloud Base Altitude Spatial Extrapolator (CBASE).  In this paper we
evaluate its performance by comparing CBASE \ensuremath{z}{} against \ensuremath{z}{} observed by
ground-based ceilometers.

The cloud base of interest in this analysis is the base of the lowest cloud in
each column. Even if CALIOP can also detect the base heights of other layers
in multilayer situations, it is the base height of the lowest cloud that is of
largest interest for many applications (e.g., surface radiation
estimates). 

Section~\ref{sec:data} of this article describes the data sources used in
determining and evaluating \ensuremath{z}{}.  In Section~\ref{sec:algorithm} we describe
the algorithm and evaluate its performance, including error statistics.  The
publicly available processed CBASE output is described in
Section~\ref{sec:results}.  We conclude in Section~\ref{sec:conclusions} with an
outlook on the longstanding questions that the CBASE dataset can address.

\section{Data}
\label{sec:data}

Two classes of data are used in this work: cloud lidar data, from which we
intend to derive a global \ensuremath{z}{} dataset; and ground-based observations used as
reference measurements of \ensuremath{z}{} to train and evaluate the algorithm by which
\ensuremath{z}{} is determined from the satellite data.

Table~\ref{tab:data} lists the URLs for all datasets used in this paper.

\subsection{CALIOP VFM}

The input satellite data to our analysis is from the Cloud--Aerosol Lidar with
Orthogoncal Polarization \citep[CALIOP,][]{Winker2007} on board the Cloud--Aerosol Lidar and Infrared Pathfinder
Satellite Observation (CALIPSO) satellite that is part of the A-Train
satellite constellation \citep{Stephens2002} on a
sun-synchronous low-Earth orbit with equator crossings at approximately 1330 hours local
time. The cloud base product relies on the retrieved vertical feature mask
\citep[VFM,][]{vaughan2002}.  For each CALIOP lidar backscatter profile, the VFM identifies features
such as clear air, cloud, aerosol, or planetary surface; this is termed the ``feature
type''.  (When the lidar beam is completely attenuated, this is reported as a
feature type.)  In addition to the feature type, the VFM records the degree of
confidence in the identification (``none'' to ``high'', termed the ``feature
type QA flag''); the thermodynamic phase of a layer identified as cloud as well
as the degree of confidence therein (termed ``ice water phase'' and ``ice water
phase QA flag''); the horizontal distance over which the algorithm had to
average to identify a feature above noise and molecular atmospheric scattering
(``horizontal averaging distance'').  

In the present analysis, we use VFM version 4.10 \citep{vfm}, the current
``standard'' release, for the years 2007 and 2008.  The VFM files are obtained
from ICARE (\url{http://www.icare.univ-lille1.fr/}).

\subsection{Airport ceilometers}



%% the next setup is needed to get the corrected combined 2008 data for
%% comparison to 2B-GEOPROF-LIDAR



For optimizing several parameters of the algorithm, for determining the expected
cloud base uncertainty, and for evaluation of the trained algorithm, reference
measurements of \ensuremath{z}{} are required.  The source of these ``true'' \ensuremath{z}{} in this work
is ground-based cloud observations at airports.  Weather observations at
airports are disseminated worldwide in aviation routine and special weather
reports \citep[METARs and SPECIs, collectively referred to as METARs
henceforth,][]{metar}.  Apart from providing airport weather information for
aviation, METAR data is used for assimilation into numerical weather prediction
(NWP) models \citep[e.g.,][]{Benjamin2016, Dee2011}.  In many locations, \ensuremath{z}{}
reported in METARs is measured by a ceilometer over a period of time (tens of
minutes) and then objectively grouped into cloud layers and their respective
fractional coverages, using the temporal variation at a fixed point under an
advected cloud field as a proxy for spatial variability of the cloud field
\citep[e.g.,][]{Heese2010}.  METAR data is widely distributed and archived; the
data for the present analysis was downloaded from the Wunderground
archive (\url{https://www.wunderground.com/history/airport/}).

In the United States, \ensuremath{z}{} is mostly derived automatically by laser ceilometers
that form part of Automated Surface Observing Stations \citep[ASOS,][]{asos}
system; see, e.g., \cite{An2017,Ikeda2017} for recent examples of ASOS
application to deriving cloud climatologies or NWP model evaluation.  In other
parts of the world, the cloud bases may be estimated by human observers or may
be omitted under certain conditions when the lowest cloud base is higher than
5000~\unit{feet}, complicating objective comparison to satellite \ensuremath{z}{}.  To ensure that
the ceilometer \ensuremath{z}{} are of high and spatially uniform quality, we restrict
ourselves to METARs from the contiguous continental United States.

There are 1645 %
stations throughout the continental USA that lie within 100~km of a CALIOP
footprint.  In normal operation, the time resolution of \ensuremath{z}{} reports is 1~h, but
during rapidly changing conditions, more frequent updates may be provided; for
comparison to satellite \ensuremath{z}{}, the ceilometer observation closest in time to the
satellite overpass is used, provided that the time difference is less than 1~h.
For training the algorithm, we use ceilometer observations from the year 2008.
For unbiased evaluation of the algorithm performance, a statistically
independent dataset is required; we use ceilometer observations from the same
stations from the year 2007.  Figure~\ref{fig:asos} shows the locations of these
stations along with the number of satellite--ceilometer \ensuremath{z}{} coincidences and the
closest co-location distance during the year 2007.

\section{CBASE algorithm development and evaluation}
\label{sec:algorithm}

The CBASE algorithm and evaluation proceed in four steps:
\begin{enumerate}
\item We determine the cloud base height from all CALIOP profiles where the
  surface generates a return, indicating that the lidar is not completely
  attenuated by cloud.  We refer to this as the \textit{column
    \ensuremath{z_\text{c}}{}} in the sense that it is local to the CALIOP column.
\item Using ground-based ceilometer data, we determine quality of cloud base
  height depending on a number of properties of the CALIOP profile.  Assuming
  those properties suffice to determine the quality of the \ensuremath{z_\text{c}}{} estimate, we
  can then predict the quality of a cloud base as a function of those factors.
  The quality metric we use is the root mean square error (RMSE); the category
  RMSE determined from comparison to ceilometer \ensuremath{z_\text{c}}{} then serves as the
  (sample) estimate of the predicted (population) standard deviation of the
  measurement error $\ensuremath{z_\text{c}} - \hat{z}$, i.e., the predicted \ensuremath{z_\text{c}}{}
  uncertainty.  We denote this column uncertainty as $\ensuremath{\sigma_\text{c}}$.  In the language
  of machine learning, we refer to this step 
  as \textit{training} the algorithm on the ceilometer data to predict \ensuremath{z_\text{c}}{} and
  $\ensuremath{\sigma_\text{c}}$.
\item Based on the predicted quality of each profile cloud base, we either reject
  the column cloud base or combine it with other cloud cloud bases within a
  distance $D_\text{max}$ of the point of interest to arrive at an 
  estimate of \ensuremath{z}{} and $\sigma$ at that point.  We refer to \ensuremath{z}{} and $\sigma$
  as the CBASE cloud base height and cloud base height uncertainty.
\item Using a statistically independent validation dataset, we verify that the
  predicted \ensuremath{z}{} and $\sigma$ are correct.
\end{enumerate}

This section is divided into four subsections, one for each algorithm step
enumerated above.

\subsection{Determination of CALIOP column \ensuremath{z}{}}
\label{sec:algorithm:local}
Profile \ensuremath{z_\text{c}}{} is determined from the CALIOP VFM for each profile with a surface
return.  The rationale is that a surface return indicates that the lidar did not
attenuate within the cloud, and that the lower limit of the layer identified as
cloud therefore corresponds to the cloud base; Figure~\ref{fig:method}
illustrates the idea.  For these profiles, the location, \ensuremath{z_\text{c}}{}, cloud top height,
feature type between the cloud base and the surface,
cloud thermodynamic phase, and associated quality assurance flags from the VFM
algorithm are recorded.

\subsection{Determination of CALIOP column cloud base quality}
\label{sec:algorithm:qual}
We assess the quality of the CALIOP \ensuremath{z_\text{c}}{} using the RMSE with respect to the
ceilometer-observed $\hat{z}$.  The RMSE is defined as
\begin{equation}
  \label{eq:rmse}
  \text{RMSE} = \sqrt{\frac{1}{N}\sum\limits_{i = 1}^{N}\left(\ensuremath{z_\text{c}}^i - \hat{z}\right)^2}.
\end{equation}
The sum runs over all CALIOP profiles containing at least one cloud layer and a
surface return that are within 100~km horizontal distance of the ceilometer,
occurred within 3600~s of a ceilometer observation, and have their lowest CALIOP
cloud feature within 3~km of the surface.  Ceilometer observations are only used
if the observation closest in time to the CALIPSO overpass contains a cloud
within 3~km of the surface.  This height limit is imposed because a subset of
the ceilometers has a range limit of 12500~feet, and all ceilometers report
ceilings above 10000~feet with reduced granularity (500~feet); the 3~km
threshold is safely below these ceilometer limitations and mimics the
International Satellite Cloud Climatology Project \citep[ISCCP,][]{Rossow1999}
definition of low cloud ($p > 680\text{ hPa}$).

The following metrics, which are useful for a qualitative assessment of the
quality of the satellite cloud base, are also calculated but play no
quantitative role in the algorithm:
\begin{description}
\item[Correlation coefficient] between the CALIOP cloud base and ground-based
  observation of the cloud base.  We use the Pearson correlation coefficient
  (ideally unity).  
\item[Linear regression slope and intercept] (ideally 1 and 0, respectively).  
\item[Retrieval bias,] defined as
  \begin{equation}
    \label{eq:bias}
    \mbox{bias} = \frac{1}{N}\sum\limits_{i = 1}^{N}\left(\ensuremath{z_\text{c}}^i - \hat{z}\right),
  \end{equation}(ideally 0)
\end{description}

CALIOP's ability to detect cloud base depends on the properties of the cloud.
Therefore, we expect that the \ensuremath{z_\text{c}}{} quality will vary between
different cloud profiles.  We expect that measuring the quality as a function of various
properties of the CALIOP column will allow us to predict the quality of other
columns with the same combination of properties.  The properties that are easily
accessible in a single column and have substantial effects on quality are:
\begin{itemize}
\item horizontal distance $D$ from the ceilometer,
\item number of column cloud bases within horizontal distance $D_\text{max}$,
\item CALIOP VFM feature quality assurance flag,
\item geometric thickness of the lowest cloud layer,
\item CALIOP thermodynamic phase determination of lowest cloud,
\item feature type, if any, detected between the lowest cloud and the surface, and
\item horizontal averaging distance required for CALIOP cloud feature
  detection.
\end{itemize}
For illustrative purposes, Figure~\ref{fig:quality-qa} and
Table~\ref{tab:quality-qa} summarize the joint distribution of CALIOP and
ceilometer \ensuremath{z_\text{c}}{} faceted by the CALIOP VFM feature quality assurance flag.

Based on determining the retrieval quality as a function of one variable at a
time (integrating over the sample distribution of the remaining variables), the
following classes of CALIOP profiles are discarded:
\begin{itemize}
\item CALIOP VFM quality assurance worse than ``high'' ,
\item ``invalid'' or ``no signal'' layers between the surface and the lowest
  cloud layer (indicating that although the surface may generate a detectable
  return, the lidar is sufficiently attenuated that the cloud base, which
  scatters less strongly than the surface, is unreliable),
\item minimum CALIOP cloud detection horizontal averaging distance within the
  lowest cloud layer greater than 1~km (indicating that, although average cloud
  properties are known at the averaging length scale, those properties may not
  be representative of the particular CALIOP footprint under consideration), or
\item thermodynamic phase of the lowest layer determined to be other than liquid
  by the CALIOP VFM algorithm (the reason for this is that not enough such
  columns exist to determine the RMSE reliably in each of the categories defined
  below).
\end{itemize}
Figure~\ref{fig:quality-qa-other-cuts} and Table~\ref{tab:quality-qa-other-cuts}
summarize the joint distribution of CALIOP profile \ensuremath{z_\text{c}}{} and ceilometer
$\hat{z}$ after these selection criteria for comparison with the unfiltered
joint distributions in Figure~\ref{fig:quality-qa}.

The remaining variables are discretized roughly into quintiles of their
distribution within the VFM dataset with the
following boundaries:
\begin{itemize}
\item horizontal distance $D$ from the ceilometer, with boundaries 0, 40, 60,
  75, 88, and 100~km (distance greater than 100~km is discarded),
\item number of CALIOP columns $n$ with a cloud layer and a surface return
  within 100~km horizontal distance from the ceilometer, with boundaries at 0,
  175, 250, 325, 400 (multiplicities greater than 400 are accepted), and
\item geometric thickness $\Delta z$ of the lowest cloud layer, with boundaries
  at 0, 0.25, 0.45, 0.625, and 1~km (thickness greater than 1~km is accepted).
\end{itemize}

We can now consider the joint distribution of CALIOP and ceilometer cloud bases
for each combination of the above variables to derive the RMSE of each
combination.  Throughout this work, we use cloud base height above ground level
(AGL); using height above mean sea level (MSL) would introduce an intrinsic
correlation between satellite and ceilometer cloud base height due to the
varying terrain height, which would lead to an unrealistically positive
assessment.  To convert cloud base heights to AGL height, we subtract the
surface elevation contained in the CALIOP VFM data files, which in turn comes
from the CloudSat R05 surface digital elevation model. 

When calculating aggregate statistics such as the RMSE, a further consideration
comes into play.  \ensuremath{z_\text{c}}{} above ground is positive-definite, which imposes a
physical phase-space boundary.  Due to this boundary, the satellite \ensuremath{z_\text{c}}{}
estimate is intrinsically biased high (negative excursions due to symmetric
random error may be removed by the phase-space boundary, but positive excursions
are not), and the bias decreases with increasing satellite \ensuremath{z_\text{c}}{} estimate
(when true \ensuremath{z_\text{c}}{} is high, it is less likely that measurement error would lead
to a negative AGL \ensuremath{z_\text{c}}{}).  Since this effect constitutes a bias rather than a
random error, it cannot be eliminated by averaging over large sample sizes, but
instead needs to be corrected for.  Since the effect is nonlinear in \ensuremath{z_\text{c}}{}, a
nonlinear correction method is required.  Our choice of nonlinear bias
correction is the support vector machine \citep[SVM;][]{Cortes1995}.  The SVM is
a machine-learning algorithm formulated to learn classification
\citep{Cortes1995} or regression \citep{Vapnik1995} tasks from a training
dataset, discarding outliers and accommodating nonlinear functions
\citep[e.g.,][]{Smola2004}.  We train an $\epsilon$-regression SVM, implemented
as an R package \citep{e1071} using the libsvm library \citep{svm}, separately
for each $D$, $n$, and $\Delta z$ category, using the 2008 ceilometer overpass
training dataset.  The correction function is not trivial to represent because
of its dependence on $z_\text{c}$, $D$, $n$, and $\Delta z$ (which can be
correlated).  To reduce the dimensionality of this multivariate correction, we
have used the training dataset (with its joint distribution of $z_\text{c}$,
$D$, $n$, and $\Delta z$) to calculate an ensemble of correction factors that
can be expected in a realistic sample of clouds, shown in
Figure~\ref{fig:svm-correction}.  The full multivariate correction function,
implemented in R, will be released to \url{https://github.com/jmuelmen}
concurrently with the publication of this manuscript.

Following bias correction, the sample RMSE is calculated for each combination of
$D$, $n$, and $\Delta z$.  The sample RMSE is taken as an estimate of the
statistical uncertainty $\ensuremath{\sigma_\text{c}}(D,n,\Delta z)$ on the CALIOP profile \ensuremath{z_\text{c}}{}.
Note that $D$ and $\Delta z$ exist for each profile, whereas $n$ is defined for
the group of suitable profiles around the point of interest.  Since the
predicted uncertainty is multivariate, it is also nontrivial to visualize.  We
again use the training dataset as an ensemble on which to perform
one-dimensional projections of $\ensuremath{\sigma_\text{c}}(D,n,\Delta z)$ onto each of the
predictor variables.  These projected $\ensuremath{\sigma_\text{c}}$ density estimates are shown in
Figure~\ref{fig:eval-uncertainty}.  The full multivariate $\ensuremath{\sigma_\text{c}}$ prediction
function, implemented in R, will be released to
\url{https://github.com/jmuelmen} concurrently with the publication of this
manuscript.

\subsection{Combination of column cloud bases}
\label{sec:algorithm:combination}
CALIOP \ensuremath{z}{} only exists sporadically, 
when CALIOP happens to hit a sufficiently thin cloud.  To infer the \ensuremath{z}{} at a
point of interest that does not necessarily coincide with the location of
a thin-cloud CALIOP column, we proceed as follows.  We first select all CALIOP
column \ensuremath{z_\text{c}}{} measurements within a horizontal distance $D_\text{max} = 100$~\unit{km} of
the point that satisfy the additional quality cuts described in
Section~\ref{sec:algorithm:qual}. 

For each remaining column $\ensuremath{z}{}_{\text{c},i}$, we determine the predicted
uncertainty $\sigma_{\text{c},i}$ based on the categories established in the previous
section.  We determine a combined \ensuremath{z}{}
\begin{equation}
  \label{eq:combo-z}
  \ensuremath{z} = \frac{\sum\limits_i^n w_i \ensuremath{z_\text{c}}^i}{\sum\limits_i^n w_i}
\end{equation}
with weights
\begin{equation}
  \label{eq:weights}
  w_i = \frac 1 {\sigma_{\text{c},i}^2}
\end{equation}
(optimal weights for uncorrelated least-squares).  The sum is calculated over
the $n$ \ensuremath{z_\text{c}}{} estimates within $D_\text{max}$ that satisfy all criteria listed
in the previous subsection.  In practice, the individual
measurements of cloud base are highly correlated with fairly similar
$\sigma_i$.  The cloud base estimate by Eq.~(\ref{eq:combo-z}) with weights
given by Eq.~(\ref{eq:weights}) remains unbiased even in the presence of
correlations.  However, for the combined cloud base uncertainty,
the uncorrelated weights would yield a biased estimate in the presence of
correlations.  The expression
\begin{equation}
  \label{eq:combo-sigma}
  \sigma^2 = \frac 1 n \sum\limits_i^n \sigma_{\text{c},i}^2
\end{equation}
yields acceptable results, as would be expected for highly correlated and fairly
similar $\sigma_{\text{c},i}$.  

\subsection{Evaluation of CBASE \ensuremath{z}{} and $\sigma$}
\label{sec:algorithm:eval}

Having trained the algorithm on data from the year 2008, we evaluate it using a
statistically independent dataset from the year 2007.  In the evaluation
dataset, the ``true'' (i.e., ceilometer-measured) $\hat{z}$ is known in
addition to the estimated \ensuremath{z}{} and the estimated cloud base uncertainty $\sigma$,
determined according to the procedure described in the previous section.
Figure~\ref{fig:eval} shows the joint distribution of CBASE \ensuremath{z}{} and
ceilometer-observed $\hat{z}$.  

For satellite-derived measurements of \ensuremath{z}{} that are unbiased with respect
to the ceilometer-observed $\hat{\ensuremath{z}}$ and have correctly estimated
uncertainties $\sigma$, the pdf of the quantity $(\ensuremath{z} - \hat{z})/\sigma$ has zero
mean and unit standard deviation. In our evaluation dataset, we find a mean of
0.04 and a standard
deviation of 1.06, shown in
Figure~\ref{fig:pull}; this corresponds to a \ensuremath{z}{} bias of %
4\% and
uncertainty bias of %
6\%,
both relative to the predicted uncertainty.  Thus, we find that both the cloud
base estimate and the uncertainty estimate are unbiased at better than the 10\%\
level.

As a further test of the reliability of the expected uncertainty, we divide the
validation dataset into deciles of the expected uncertainty.
Table~\ref{tab:rmseclass} shows that the actual RMSE within each decile is
within 10\% of the expected uncertainty (with the exception of the highest-uncertainty
decile) and that linear regressions within each
decile are close to the one-to-one line.

To check that the algorithm satisfies its design constraints (i.e., to ensure
that we made no methodological when implementing the algorithm), we have also
verified that linear regression between $z$ and $\hat{z}$ has zero intercept and
unit slope, and that the quantity $(z - \hat{z})/\sigma$ has zero mean and unit
standard deviation, when this validation is performed on the training dataset.

It is possible that \ensuremath{z}{} estimates outside North America could have greater
biases or greater uncertainty than this evaluation leads us to believe.  This
would be the case if continental clouds over North America are not
representative of clouds elsewhere in a way that is not accounted for by the
cloud properties considered by the uncertainty estimate.  Since the validation
sample spans an entire year on a continental scale, we expect that most cloud
morphologies are included.
However, cloud types that occur predominantly over ocean, namely marine stratocumulus with
horizontally extensive but vertically thin liquid-phase anvils, 
present a particular challenge to the method.  Due to the
typical \ensuremath{z}{} uncertainty of several hundred m, the method is unlikely to be
applied to stratocumulus cloud; nevertheless, a marine-cloud validation dataset
would be desirable.  For the present work, no suitable marine-cloud evaluation
dataset was available; ship-based \ensuremath{z}{} observations were either based on human
observers with coarse vertical resolution and a precision that is difficult to
characterize; or available only over a limited duration at limited
locations, resulting in a severely statistics-limited set of coincidences with
the CALIOP track.

\subsection{Comparative evaluation of CBASE and 2B-GEOPROF-LIDAR}

Comparison with 2B-GEOPROF-LIDAR cloud bases (version P2\_R04\_E02, based on the
2B-GEOPROF and CALIOP VFM products) is shown in Figure~\ref{fig:eval-2b}.
2B-GEOPROF-LIDAR distinguishes between radar-only, lidar-only, and radar--lidar
combined cloud bases; the latter category is rare for warm clouds and is not
shown.  For radar-only clouds, the mean error is large because the radar \ensuremath{z}{}
predominantly clusters around the top of the ground clutter region with little
dependence on the actual \ensuremath{z}{}.  

Lidar-only 2B-GEOPROF-LIDAR cloud base performs comparably to the CBASE cloud
base on average; this is to be expected, as the underlying physical measurement
(the CALIOP attenuated backscatter) is the same for all three products
considered (2B-GEOPROF-LIDAR, CALIOP VFM, and CBASE).
Figure~\ref{fig:comp-2b-cbase} shows the relationship between CBASE \ensuremath{z}{} and
the 2B-GEOPROF-LIDAR cloud base closest to the ceilometer for each overpass.
The CBASE \ensuremath{z}{} for low clouds tends to be higher than the 2B-GEOPROF-LIDAR
estimate because the CBASE algorithm has been designed to agree with ceilometer
heights, which also tend to be higher than the 2B-GEOPROF-LIDAR estimate (see
Figure~\ref{fig:eval-2b}).  Otherwise, the relationship is fairly close (linear
correlation coefficient of 0.79), again as expected due to the similarity in the
underlying measurement.

Unlike 2B-GEOPROF-LIDAR and the CALIOP VFM, CBASE provides a validated
point-by-point uncertainty estimate, which allows an analysis to select only
low-uncertainty cases or to statistically weight \ensuremath{z}{} according to
uncertainty, as appropriate for the application.

\section{Results and data product availability}
\label{sec:results}

Geographic distributions of the mean \ensuremath{z}{} are shown for daytime and nighttime
Calipso overpasses in Figure~\ref{fig:geo}.  Over most of the globe, especially
over land, daytime \ensuremath{z}{} is higher than nighttime \ensuremath{z}{}, consistent with the
diurnal deepening of the planetary boundary layer.
Figures~\ref{fig:uncertainty} and \ref{fig:uncert-quantiles} show the
distribution of \ensuremath{z}{} uncertainties.  A larger fraction of nighttime cloud
bases falls into the lowest uncertainty range (200 to 350~\unit{m}), while the
the nighttime uncertainty distribution peaks slightly higher than the daytime
uncertainty distribution and features a substantial tail above 500~\unit{m} that
is not present in the daytime distribution.  CALIOP benefits from higher signal
to noise ratio during nighttime, which may lead to lower $\sigma$, but this
effect would be convoluted with potential differences between daytime and
nighttime clouds that can lead to different \ensuremath{z}{} uncertainties.  Training a
potential future update of the algorithm on daytime and nighttime profiles
separately may reduce $\sigma$.

As an example application, we consider the surface downwelling longwave
radiation \ensuremath{F_\text{surf}^\downarrow}{}, which is strongly affected by cloud base temperature.
\cite{Henderson2013} derive a global \ensuremath{F_\text{surf}^\downarrow}{} sensitivity to \ensuremath{z}{} of
1.5~\unit{W~m^{-2}} for a \ensuremath{z}{} perturbation of one CloudSat height bin
(240~\unit{m}); as Table~\ref{tab:2bgeoprof} and Figure~\ref{fig:eval-2b} show,
the CloudSat $\sigma$ specifically for the low clouds at the focus of the
present work is likely greater than 240~\unit{m}, which corroborates the
480~\unit{m} uncertainty estimate of \cite{Kato2011}. To arrive at a
conservative estimate of the improvement in \ensuremath{F_\text{surf}^\downarrow}{} uncertainty that might be
possible by utilizing the CBASE predicted $\sigma$, we compare two \ensuremath{F_\text{surf}^\downarrow}{}
uncertainty distributions: one based on a globally constant 400~\unit{m} $\sigma$
(Figure~\ref{fig:dlr}a) and one with the CBASE $\sigma$ achievable by selecting
the highest-quality percentile of the CBASE dataset
(Figure~\ref{fig:dlr}b). This selection provides a $\sigma$ of approximately
250~\unit{m} in the extratropics as well as the nighttime tropical continents
and SCu regions, and approximately 400~\unit{m} throughout the tropics during
daytime, according to Figure~\ref{fig:uncert-quantiles}.  Globally, the \ensuremath{F_\text{surf}^\downarrow}{}
uncertainty is reduced from 3.1~\unit{W~m^{-2}} to 1.8~\unit{W~m^{-2}}, assuming
that the \ensuremath{z}{} uncertainty contribution to the \ensuremath{F_\text{surf}^\downarrow}{} uncertainty is dominated
by low clouds.  Improvements are especially large in the marine stratocumulus
regions and the extratropical oceans, where extensive low cloud often overlies
cool air with relatively low longwave emission by water vapor. The selection
reduces the available statistics by a factor of 100, but analyses based on
A-Train data are usually not statistics-limited.

The CBASE \ensuremath{z}{} and $\sigma$ dataset \citep{cbase} spanning the years 2007 and
2008 is freely available at Deutsches Klimarechenzentrum (DKRZ) under the DOI
\url{https://doi.org/10.1594/WDCC/CBASE}.  The dataset is provided in two
spatial resolutions corresponding to different window sizes within which CALIOP
profiles are combined: $D_\text{max} = 40$~\unit{km} and
$D_\text{max} = 100$~\unit{km}.  CBASE provides two files for each CALIOP VFM
input file: one using a 40 km window to detect the cloud field base height, and
one using a 100 km window. (The input CALIOP VFM dataset is organized by the
daytime (D)/nighttime (N) half of each orbit.) The file name pattern is
\verb+CBASE-{40|100}.<date>T<time>{D|N}.nc+ (identical to the input CALIOP VFM
file name with the exception of the product name and file-type extension). Files
are organized into subdirectories by half-orbit start date.  In case no cloud
base heights are detected within a half-orbit, no output file is
produced. Otherwise, each CALIOP VFM input file results in a 40 km-resolution
and a 100-km resolution CBASE file. The measurement quality is reported as a
quantitative uncertainty estimate for each cloud field.

Concurrently with publication of the final paper, the
source code used to produce the dataset and evaluation plots will be released
at \url{https://github.com/jmuelmen}.

\conclusions
\label{sec:conclusions}

We have presented the CBASE algorithm, which derives the cloud base height \ensuremath{z}{} from CALIOP lidar
profiles.  This algorithm produces \ensuremath{z}{} not only for thin clouds but also for
clouds thick enough to attenuate the lidar (optical thickness $\tau \gtrsim 5$),
based on the assumed mesoscale homogeneity of cloud base height within an
airmass.  In addition to the \ensuremath{z}{} estimate, the CBASE algorithm supplies an
expected uncertainty $\sigma$ on the \ensuremath{z}{}.  The CBASE dataset is available for the years 2007
and 2008 at \url{https://doi.org/10.1594/WDCC/CBASE}.

CBASE \ensuremath{z}{} and $\sigma$ have been evaluated using
ground-based airport ceilometers over the contiguous United States using a
data sample unbiased by the training of the algorithm.  The evaluation showed that
\ensuremath{z}{} and $\sigma$ are unbiased at the better
than 10\% level: the bias on the \ensuremath{z}{} is %
4\%,
and the bias on the uncertainty is %
6\%, both relative to the expected uncertainty.

The performance of CBASE \ensuremath{z}{} is similar to that of 2B-GEOPROF-LIDAR
lidar-only \ensuremath{z}{} when validated against the same collocated ceilometer measurements, which is based on the same underlying physical measurement.
However, the validated \ensuremath{z}{} uncertainty provided by CBASE allows for selection
of only accurate cloud base heights or for statistically weighting of \ensuremath{z}{}
according to expected uncertainty.  This, in turn, makes the CBASE \ensuremath{z}{} useful
for pressing problems in climate research that require accurate knowledge of
cloud geometry, such as surface downwelling longwave radiation or cloud
subadiabaticity, which will be presented in future work.

\begin{acknowledgements}
  We thank Patric Seifert and Albert Ansmann for valuable suggestions on the
  algorithm; the editor and two anonymous reviewers for comments that have
  improved the manuscript; ICARE for hosting the CALIOP VFM dataset, which was
  originally obtained from the NASA Langley Research Center Atmospheric Science
  Data Center; DKRZ for computing and data hosting; and the R Foundation for
  Statistical Computing for providing the open-source software used for this
  analysis \citep{R}.  This research was funded by the European Union under ERC Starting
  Grant QUAERERE, grant agreement 306284, and by the United States National
  Science Foundation under grant agreements AGS-1013423 and AGS-1048995.
\end{acknowledgements}

\bibliographystyle{copernicus}
\bibliography{method.bib}

\clearpage

\begin{table}
  \centering
  \caption{Data sources used in this analysis}
  \begin{tabular}{ll}
    \hline\hline
    Data product & URL  \\\hline
    CALIOP VFM & \url{http://www.icare.univ-lille1.fr/archive?dir=CALIOP/VFM.v4.10/} \\
    ASOS locations & \url{http://www.rap.ucar.edu/weather/surface/stations.txt}\\
    METAR data & \url{https://www.wunderground.com/history/airport/}\footnotemark[1] \\
    CBASE & \url{https://doi.org/10.1594/WDCC/CBASE} \\
    \hline\hline
  \end{tabular}
  \belowtable{\footnotemark[1] As a first step, ASOS station identifiers within 100~\unit{km}
    great-circle distance of a CALIOP footprint are identified; as a second
    step, the ICAO identifier of the ASOS station is then used to query the
    Wunderground METAR database.}
  \label{tab:data}
\end{table}
\begin{figure}
{\centering \includegraphics[width=0.95\textwidth]{f01} 

}
  \caption{ASOS ceilometers used for CBASE \ensuremath{z}{} evaluation.  The size of the
    marker indicates the number of satellite--ceilometer \ensuremath{z}{} coincidences during
    the year 2007.  Color indicates the closest co-location distance achieved in
    2007.}
  \label{fig:asos}
\end{figure}

\begin{figure}
  \centering
% Note to self: for figure renaming, run:
%  for i in {1..14} ; do ln -s $(head -n $((i + 1)) files.txt | tail -n 1 | sed -e 's/.*\(figure.*\)}/\1.pdf/') $(printf f%02d.pdf $i); done
  \includegraphics[width=0.5\linewidth,keepaspectratio=true]{f02}
  \caption{Schematic of CALIOP cloud base determination and evaluation strategy.
    In optically thick clouds (left and center), the lidar attenuates
    significantly within the cloud, rendering the cloud base information
    unreliable.  However, \ensuremath{z}{} of thin clouds (right) can be used as a proxy
    for thick clouds in a cloud field with homogeneous \ensuremath{z}{}.}
  \label{fig:method}
\end{figure}

\begin{figure*}
{\centering \includegraphics[width=0.95\textwidth]{f03} 

}
  \caption{Scatter plots of CALIOP versus ceilometer cloud base height faceted
    by the CALIOP VFM QA flag; all CALIOP profiles meeting the temporal and
    spatial collocation requirements with a METAR enter into this plot.  Color
    indicates the number of CALIOP profiles within each bin of ceilometer and
    CALIOP \ensuremath{z}{}; black lines are contours of the empirical joint probability
    density; the red line is a linear least-squares fit, with 95\% confidence
    interval shaded in light red; the blue line is a generalized additive model
    regression \citep{Wood2011}, with 95\% confidence interval shaded in light
    blue (due to the large data set, the line width exceeds the confidence
    intervals in these plots); the dashed gray line is the one-to-one line.
    Statistics of the relationship between CALIOP and ceilometer base heights
    are provided in Table~\ref{tab:quality-qa}.}
  \label{fig:quality-qa}
\end{figure*}

\begin{table*}
  \centering
  \caption{Statistics of the relationship between ceilometer and CALIOP cloud
    base height faceted by CALIOP VFM QA flag.  Shown are the number of CALIOP
    profiles $n$, the product-moment correlation coefficient $r$ between CALIOP
    and ceilometer \ensuremath{z}{}, the RMSE, bias, and linear least-squares
    fit parameters.}
  \label{tab:quality-qa}
% latex table generated in R 3.4.4 by xtable 1.8-2 package
% Mon Nov  5 16:41:56 2018
\begin{tabular}{lrrrrl}
  \hline
\hline
QA flag & $n$ & $r$ & RMSE (m) & bias (m) & fit \\ 
  \hline
none & 1410553 & 0.192 & $1.05 \times 10^{3}$ & $-$471. & $\hat{z} = 0.193 z + \ensuremath{1.03 \times 10^{3}}$ m \\ 
  low & 301250 & 0.471 & 710. & $-$115. & $\hat{z} = 0.456 z + 650.$ m \\ 
  medium & 212723 & 0.502 & 707. & $-$77.1 & $\hat{z} = 0.476 z + 602.$ m \\ 
  high & 2877967 & 0.554 & 629. & 9.85 & $\hat{z} = 0.526 z + 485.$ m \\ 
   \hline
\hline
\end{tabular}

\end{table*}

\begin{figure*}
{\centering \includegraphics[width=0.95\textwidth]{f04}

}
  \caption{As in Figure~\ref{fig:quality-qa}, but applying all 
    requirements listed in Section~\ref{sec:algorithm:combination}.}
  \label{fig:quality-qa-other-cuts}
\end{figure*}

\begin{table*}
  \centering
  \caption{As in Table \ref{tab:quality-qa}, but applying all 
    requirements listed in Section~\ref{sec:algorithm:combination}.}
  \label{tab:quality-qa-other-cuts}
% latex table generated in R 3.4.4 by xtable 1.8-2 package
% Mon Nov  5 16:41:56 2018
\begin{tabular}{lrrrrl}
  \hline
\hline
QA flag & $n$ & $r$ & RMSE (m) & bias (m) & fit \\ 
  \hline
none & 189554 & 0.573 & 635. & $-$77.3 & $\hat{z} = 0.557 z + 549.$ m \\ 
  low & 177058 & 0.566 & 634. & $-$154. & $\hat{z} = 0.556 z + 567.$ m \\ 
  medium & 135943 & 0.600 & 615. & $-$113. & $\hat{z} = 0.587 z + 511.$ m \\ 
  high & 2136337 & 0.624 & 577. & $-$36.8 & $\hat{z} = 0.581 z + 470.$ m \\ 
   \hline
\hline
\end{tabular}

\end{table*}

\begin{figure*}
{\centering \includegraphics[width=0.95\textwidth]{f05}

}
  \caption{Density estimates of the projection of the SVM correction function.
    The training dataset (ceilometer overpasses in 2008) is used as the ensemble
    for performing the projection.}
  \label{fig:svm-correction}
\end{figure*}

\begin{figure*}
{\centering \includegraphics[width=0.95\textwidth]{f06}

}
  \caption{Density estimates of the projection of
    $\ensuremath{\sigma_\text{c}}(D, n, \Delta z)$ onto each of the uncertainty
    predictor variables.  The training dataset (ceilometer overpasses in 2008)
    is used as the ensemble for performing the projection.}
  \label{fig:eval-uncertainty}
\end{figure*}

\begin{figure*}
{\centering \includegraphics[width=0.95\textwidth]{f07}

}
  \caption{Scatter plot of CBASE versus ceilometer \ensuremath{z}{} for all A-Train
    overpasses over the CONUS available for 2007; for description of the
  plot elements, see Figure~\ref{fig:quality-qa}.  The linear fit has slope
  0.98 and intercept $33.96$~\unit{m}.}
  \label{fig:eval}
\end{figure*}

\begin{figure}
{\centering \includegraphics[width=0.5\textwidth]{f08}

}
  \caption{Distribution function of cloud base error divided by predicted
    uncertainty; for the ideal case of unbiased \ensuremath{z}{} and unbiased
    uncertainty, the distribution would be Gaussian with zero mean and unit
    standard deviation.  The superimposed least-squares Gaussian fit (blue line)
    has mean 0.04 and
    standard deviation 1.06.}
  \label{fig:pull}
\end{figure}

\begin{table*}[t]
  \centering
  \caption{CBASE cloud base statistics by decile of predicted uncertainty; see
    Table~\ref{tab:quality-qa} for a description of the 
    statistics provided.}
  \label{tab:rmseclass}

% latex table generated in R 3.4.3 by xtable 1.8-2 package
% Mon Jan 29 18:26:41 2018
\begin{tabular}{lrrrrl}
  \hline
\hline
Predicted $\sigma$ (m) & $n$ & $r$ & RMSE (m) & bias (m) & fit \\ 
  \hline
(167,427] & 2624 & 0.741 & 404. & $-$46.9 & $\hat{z} = 1.03 z + 28.0$ m \\ 
  (427,453] & 2624 & 0.719 & 429. & $-$28.4 & $\hat{z} = 1.06 z - 32.0$ m \\ 
  (453,469] & 2624 & 0.703 & 461. & $-$18.8 & $\hat{z} = 1.09 z - 87.7$ m \\ 
  (469,484] & 2624 & 0.685 & 463. & $-$17.8 & $\hat{z} = 1.03 z - 18.3$ m \\ 
  (484,497] & 2624 & 0.628 & 506. & $-$6.06 & $\hat{z} = 0.976 z + 33.4$ m \\ 
  (497,508] & 2624 & 0.574 & 547. & $-$8.73 & $\hat{z} = 0.986 z + 25.5$ m \\ 
  (508,522] & 2624 & 0.596 & 547. & $-$14.1 & $\hat{z} = 1.01 z + 5.37$ m \\ 
  (522,541] & 2624 & 0.572 & 562. & $-$9.26 & $\hat{z} = 0.967 z + 49.6$ m \\ 
  (541,573] & 2624 & 0.502 & 639. & $-$22.7 & $\hat{z} = 0.939 z + 96.8$ m \\ 
  (573,748] & 2624 & 0.447 & 720. & 7.36 & $\hat{z} = 0.829 z + 197.$ m \\ 
   \hline
\hline
\end{tabular}

\end{table*}

\begin{figure*}
{\centering \includegraphics[width=0.95\textwidth]{f09}

}
  \caption{Scatter plot of 2B-GEOPROF-LIDAR versus ceilometer \ensuremath{z}{}
    faceted by the source of the cloud base (radar-only or lidar-only; due to
    their rare occurrence, combined radar--lidar base heights are not shown).
    For description of the plot elements, see Figure~\ref{fig:quality-qa}.  Statistics of the
    relationship between 2B-GEOPROF-LIDAR and ceilometer base heights are provided in
    Table~\ref{tab:2bgeoprof}.}
  \label{fig:eval-2b}
\end{figure*}

\begin{table*}
  \caption{Statistics of the relationship between ceilometer and
    2B-GEOPROF-LIDAR \ensuremath{z}{}; see Table~\ref{tab:quality-qa} for a
    description of the statistics provided.}
  \label{tab:2bgeoprof}
  \centering
% latex table generated in R 3.4.4 by xtable 1.8-2 package
% Thu Aug  9 17:04:17 2018
\begin{tabular}{lrrrrl}
  \hline
\hline
Base type & $n$ & $r$ & RMSE (m) & bias (m) & fit \\ 
  \hline
Radar & 15061 & 0.265 & 782. & 98.1 & $\hat{z} = 0.461 z + 466.$ m \\ 
  Lidar & 12813 & 0.564 & 594. & 16.3 & $\hat{z} = 0.555 z + 399.$ m \\ 
   \hline
\hline
\end{tabular}

\end{table*}

\begin{figure*}
{\centering \includegraphics[width=0.95\textwidth]{f10}

}
  \caption{Scatter plot of 2B-GEOPROF-LIDAR lidar-only versus CBASE \ensuremath{z}{}.  For
    description of the plot elements, see Figure~\ref{fig:quality-qa}; because
    both cloud base measures have comparable uncertainty, linear regression is a
    misleading diagnostic \citep{Pitkanen2016} and has not been included.  The mean
    difference between 2B-GEOPROF-LIDAR and CBASE is 0.05~\unit{km},
    the root mean square difference is 0.41~\unit{km}, and the
    correlation coefficient is 0.79.}
  \label{fig:comp-2b-cbase}
\end{figure*}



\begin{figure*}
{\centering \includegraphics[width=0.95\textwidth]{f11}

}
  \caption{Geographic distribution of mean \ensuremath{z}{} above ground
    level.  Statistics are calculated within each $5\degree\times 5\degree$
    latitude--longitude box, and separately for CALIOP daytime (top) and
  nighttime (bottom)
    overpasses.}
  \label{fig:geo}
\end{figure*}

\begin{figure}
{\centering \includegraphics[width=0.5\textwidth]{f12}

}
  \caption{Distribution of predicted \ensuremath{z}{} uncertainty $\sigma$.}
  \label{fig:uncertainty}
\end{figure}

\begin{figure*}
{\centering \includegraphics[width=0.95\textwidth]{f13}

}
  \caption{Cloud base uncertainty quantiles.  Statistics are calculated within
    each $5\degree\times 5\degree$ latitude--longitude box.  The left (right)
    column shows statistics of daytime (nighttime) retrievals; daytime and
    nighttime are defined by the CALIOP VFM product.}
  \label{fig:uncert-quantiles}
\end{figure*}

\begin{figure}
  {\centering \includegraphics[width=0.5\textwidth]{f14}
    
  }
  \caption{Uncertainty on the surface downwelling longwave radiation \ensuremath{F_\text{surf}^\downarrow}{}
    under two assumptions on \ensuremath{z}{} uncertainty: (a) constant 400~\unit{m}
    uncertainty globally and (b) uncertainty achievable by selecting a 
    high-quality subset of CBASE \ensuremath{z}.}
  \label{fig:dlr}
\end{figure}

%% <<glorious-victory,cache=FALSE,echo=FALSE>>=
%% @ 

\end{document}
